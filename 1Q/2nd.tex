\section{関連研究}

日本にはかつてゲーム業界を牽引する会社が多く存在していたが,厳しい国際競争にさらされた結果,日本の会社のゲーム開発力低下も叫ばれている.
%https://game-creators.jp/media/career/085/


若原らは、インターネット上でゲームに興味を持つ人達に有効なコミュニティサイトを実現するため、ゲームソフトの開発を目指すゲームクリエイターやゲーム初心者やゲーム愛好者を主な対象に、ゲームコミュニティ支援手法について述べた\cite{若原俊彦2009}。まず,SNSを用いたゲームコミュニティ支援システムについて述べ、このプロトタイプを構築して実験を行い、コミュニティ内のコミュニケーション機能を評価し分析した結果を報告した.特に、ゲームコミュニティを活性化する支援手法として、作品投稿機能、作品評価機能、作品検索機能、作品収集・表示機能について述べ、その特性および効果について述べた。 

馬場は,ゲーム制作プロセスの中で,ゲームの面白さを受け手にどのように伝えようとしているか,ゲームコンテンツの送り手の立場から解説している\cite{馬場哲治2005}.
%https://www.jstage.jst.go.jp/article/iieej/34/2/34_2_167/_pdf

三宅は,急速に進歩するディジタルゲームにおける人工知能の全
体像について概要説明を行った\cite{三宅陽一郎2015}.
%https://www.jstage.jst.go.jp/article/jjsai/30/1/30_45/_pdf/-char/ja