\section{緒言}

 近年、高齢者の健康維持としてeスポーツが注目されている。シルバーeスポーツ協会も発足しており、シルバーeスポーツ大会も開催され、盛り上がりを見せている。デジタルゲームと認知機能に関する効果をまとめた論文によると、空間回転能力、視覚的な注意力、視覚にかかわる短期記憶、コントラスト感度、といった認知機能を向上される効果があると示唆されている。
一方で、ゲームに対する一般的なイメージは未だに否定的だと言える。実際に香川県ではネット・ゲーム依存症対策条例が定められ、令和2年4月1日から施行されている。この条例によると、保護者は18歳未満の子どもに対して、18歳未満の子どものネット・ゲーム依存症につながるようなコンピュータゲームの利用は1日当たり60分まで(学校等休業日は90分)を上限として、義務教育終了前の子どもについては午後9時まで、それ以外は午後10時までに使用をやめることを目安に、使用に関するルールを作り遵守させる責務があるとされている。また、一般的にゲームはただの遊び、勉強の邪魔、暴力的、引きこもり、社会性の欠如という風潮があることは否めない。\\
 このように、世の中で新たな盛り上がりを見せているeスポーツだが、教育としてはまだまだ否定的だと言える。
そこで本研究ではeスポーツや教育といった論文などを踏まえて、eスポーツの教育的な有用性と危険性の両面について明らかにすることを目的とする。

%\cite{Fukuyasu2020}
%\cite{published_papers/16690353}
%\cite{Dentsu2019}
