\documentclass[resume]{hitsotsuron} %文書のクラスファイルを指定.中間発表は[chukan]

%題目の各文字列を指定
\title{ゲームクリエイターになるために必要なこと}

\papertype{令和4年度卒業研究1Q成果報告}
\author{西村伊織}
\snumber{BL19073}
\advisor{松本慎平}
\date{2022年6月3日}

\begin{document} %文書本体の開始

%題目を実際に作成
\twocolumn[%
\maketitle
]

\if0

Latexの利用に関する参考URL)

LaTeX - コマンド一覧
http://www1.kiy.jp/~yoka/LaTeX/latex.html

LaTeXコマンドシート一覧
https://members.tripod.com/e_luw/gakko/latex1/l_list.html

\fi

% 緒言
\section{緒言}

情報系専門分野においては,将来の進路としてゲームクリエイタを強く希望する大学生が多く存在する.一方で,ゲームクリエイタになるという目標の達成は容易ではない.その理由は,大学関係者が知る限り,指導学生がプロのゲームクリエイタになった事例は極めて希であるという事実による.一例であるが,広島工業大学情報学部から新卒でゲームクリエイタになった学生は,大学関係者が知る限りこれまでに存在していない.ゲーム制作を業務の一部で行う会社に就職した学生はまれに存在するが,ゲームクリエイタの輩出は未だ実現できていない.広島工業大学だけではなく,他大学でも同様である.大阪電気通信大学など,ゲーム関連会社と強い関係を持つ教育機関であっても,その数はごく少数である.ゲームクリエイタの養成に特化した専門学校であっても,新卒でゲームクリエイタの夢を勝ち取れる学生は限られる.このように,ゲームクリエイタの排出は教育機関にとって特に厳しい目標であると言わざるを得ない.しかし,今後もゲームクリエイタを目指す若者が現れることが予想される.そのため,教育機関においてゲームクリエイタの排出が従来困難であった原因について,また,ゲームクリエイタの輩出の実現に必要な取り組みや教育法について,詳しく調査することは意義がある取り組みだと言える.そこで本研究では,ゲームクリエイタになるために必要となる各素養やそれぞれの程度,これら素養の獲得に効果的な方法を明確にし,従来教育機関でゲームクリエイタの輩出が困難であった理由や,ゲームクリエイタの輩出に必要と考えられる教育的活動を明らかにすることを目的とする.

% 関連研究
\section{関連研究}

日本にはかつてゲーム業界を牽引する会社が多く存在していたが,厳しい国際競争にさらされた結果,日本の会社のゲーム開発力低下も叫ばれている.
%https://game-creators.jp/media/career/085/


若原らは、インターネット上でゲームに興味を持つ人達に有効なコミュニティサイトを実現するため、ゲームソフトの開発を目指すゲームクリエイターやゲーム初心者やゲーム愛好者を主な対象に、ゲームコミュニティ支援手法について述べた\cite{若原俊彦2009}。まず,SNSを用いたゲームコミュニティ支援システムについて述べ、このプロトタイプを構築して実験を行い、コミュニティ内のコミュニケーション機能を評価し分析した結果を報告した.特に、ゲームコミュニティを活性化する支援手法として、作品投稿機能、作品評価機能、作品検索機能、作品収集・表示機能について述べ、その特性および効果について述べた。 

馬場は,ゲーム制作プロセスの中で,ゲームの面白さを受け手にどのように伝えようとしているか,ゲームコンテンツの送り手の立場から解説している\cite{馬場哲治2005}.
%https://www.jstage.jst.go.jp/article/iieej/34/2/34_2_167/_pdf

三宅は,急速に進歩するディジタルゲームにおける人工知能の全
体像について概要説明を行った\cite{三宅陽一郎2015}.
%https://www.jstage.jst.go.jp/article/jjsai/30/1/30_45/_pdf/-char/ja

% 提案
\section{現状}

広島工業大学情報学部M教授は,前任校も含めて,これまで約150名の学生を自身の研究室や共同研究を通じて指導してきた.そのうち,ゲーム開発を業務の一環として行う会社に就職した学生は,教員の記憶の限り4名である.約3\%の学生.1名はプログラミングコンテストの実績が認められた.1名は.ACM SIGGRAPHの研究業績が認められ,内定を得た.1名は,大学時代の同人活動が認められ,内定を得た.1名は,地域でのボランティア活動が認められ,地場のゲーム会社から内定を得た.ただし,当該学生はゲーム開発部門の採用ではない.

%実験方法
\input{4th.tex}

%結論
\section{結言}

本研究では,ゲームクリエイターになるために必要となる各素養やそれぞれの程度,これら素養の獲得に効果的な方法を明確にすることを目的とした.

\bibliographystyle{junsrt}    % スタイル(bstファイル名)を記述する.jplain,plainはpLaTeXに標準で付いている.
%\nocite{*}                 % 本文中で参照していない文献をリストに載せたい場合に用いる.*を指定すると全て載せる.通常使わないのでコメントアウトしている.
\bibliography{reflist}    % bibファイル名を指定する.拡張子は除く.
\end{document} 