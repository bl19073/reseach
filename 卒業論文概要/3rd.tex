\section{eスポーツの教育的役割}

本研究では,eスポーツが教育的な役割を得るため,eスポーツによって得られる効果などに関しての論文を挙げ考察する.また,好ましい効果のみではなく,懸念される効果についても考察する.

\subsection{チームワーク能力向上}

オンライン環境下で低下するチームワーク能力を補完するトレーニング方法が提案されている\cite{Fukuyasu2020}\\.オンライン・対面の両環境でチームワーク能力の差異をアンケート調査した結果,オンライン環境では特にバックアップ能力とモニタリング能力の低下が顕著であった.これを踏まえ,オンライン環境でチームワーク能力を補完する方法として「タクティカルシューターゲーム」のVALORANTを活用したトレーニング方法を考案した.%トレーニングの実施前後でのチームワーク能力の差をt検定で確認した結果,チームワーク能力の28項目において有意差が見られた.
この論文から,チームワーク能力が重要なeスポーツを行うことで,チームワーク能力の向上が期待できることが示された.

\subsection{スマートドラッグに対する大学生の認識}

健常者の認知能力の亢進(認知的エンハンスメント)は21世紀に入って世界的な関心を集めており,最も普及の進んでいるツールが,いわゆるスマートドラッグ(賢くなる薬)である.%厚生労働省でもスマートドラッグに対する問題提起がなされているが,その定義や範囲については明確ではない.
スマートドラッグの健常者に対する影響や長期的な服用による副作用,依存性についての研究は十分とは言えない.%そのため,青少年が安易に使用することに対して警告を鳴らす必要性がある.また,海外ではサプリメントなど健康補助食品も含めてヌートロピックと呼ばれることもあるが,日本ではスマートドラッグという通称名が浸透しているため,こちらの呼称で統一する.
欧米諸国では,主に集中力を要する職種での作業効率及び能力向上のためや,大学生が試験前に学習能力を上げる目的でスマートドラッグを使用する例が増加している.また,eスポーツにおいてもスマートドラッグを使用する事例が見られる.

スマートドラッグに対する大学生の認識を調査した研究によると\cite{published_papers/16690353},大学生100名に11項目のアンケートを実施した結果,スマートドラッグについての認識は低く,普及していないことが明らかになった.%しかし,試験勉強対策にカフェイン製品を利用する学生が半数であった.欧米諸国の大学生に対する調査では
カフェイン製品やビタミン剤などを利用する学生はスマートドラッグの使用に移行する傾向が示唆されており,将来的な普及の可能性が懸念される.

この論文から,認知能力の亢進を人工的に行うことが可能であり,その薬物を恒常的に使用する可能性があることが分かった.更に,その薬物は安全性が確保されておらず,青少年への使用は安全とは言い切れない.また,
世界最大のe-Sportsリーグ大会サイトESL(Electoronic Sports League)は2015年に開催した大会で薬物検査を実施した.このことから,通常のスポーツの大会のように,薬物によるパフォーマンスの向上(ドーピング)は認められない流れになりつつある.精神的に不安定になりやすい青少年のスマートドラッグの使用には,特に注意するべきだと考えられる.

\subsection{eスポーツに関するポジティブ効果検証}

萩原らは,eスポーツ活動の前後での認知的スキルと脳波から抽出される集中度の関連について調査した\cite{Dentsu2019}.実験は男性大学生20名を対象とし,脳波の検出にはバンドタイプの簡易脳波計を使用した.eスポーツ課題を実施した前後で,一致課題と不一致課題のどちらも回答時間の減少が確認された.また,β波帯域パワーとSMR帯域パワーの増加が見られ,集中力が向上したことが確認された.なお,実験でレース型eスポーツ課題として任天堂のマリオカート8デラックスを利用し,コースはマリオカートスタジアム,モードはタイムアタック,クラスは150ccでキャラクターは対象者が選択した.

この論文から,レース型eスポーツには集中力が向上する効果を期待できることが分かった.この実験はレース型eスポーツのみで行われたが,レース型以外の種目のeスポーツでも集中力は求められるため,レース型以外のeスポーツでも集中力が高まる可能性は高いと考えられる.
