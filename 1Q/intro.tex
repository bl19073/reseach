\section{緒言}

情報系専門分野においては,将来の進路としてゲームクリエイタを強く希望する大学生が多く存在する.一方で,ゲームクリエイタになるという目標の達成は容易ではない.その理由は,大学関係者が知る限り,指導学生がプロのゲームクリエイタになった事例は極めて希であるからである.一例であるが,広島工業大学情報学部から新卒でゲームクリエイタになった学生は,大学関係者が知る限りこれまでに存在していない.ゲーム制作を業務の一部で行う会社に就職した学生はまれに存在するが,ゲームクリエイタの輩出は未だ実現できていない.広島工業大学だけではなく,他大学でも同様である.大阪電気通信大学など,ゲーム関連会社と強い関係を持つ教育機関であっても,その数はごく少数である.ゲームクリエイタの養成に特化した専門学校であっても,新卒でゲームクリエイタの夢を勝ち取れる学生は限られる.このように,ゲームクリエイタの排出は教育機関にとって特に厳しい目標であると言わざるを得ない.しかし,今後もゲームクリエイタを目指す若者が現れることが予想される.そのため,教育機関においてゲームクリエイタの排出が従来困難であった原因について,また,ゲームクリエイタの輩出の実現に必要な取り組みや教育法について,詳しく調査することは意義がある取り組みだと言える.そこで本研究では,ゲームクリエイタになるために必要となる各素養やそれぞれの程度,これら素養の獲得に効果的な方法を明確にし,従来教育機関でゲームクリエイタの輩出が困難であった理由や,ゲームクリエイタの輩出に必要と考えられる教育的活動を明らかにすることを目的とする.